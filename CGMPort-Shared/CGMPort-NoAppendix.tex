\documentclass[./CGMPort.tex]{subfiles}% LaTeX path to the root directory of the current project
\providecommand{\econtexRoot}{}
\renewcommand{\econtexRoot}{.}

% Different execution depending on whether compiling main file or as standalone subfile
\onlyinsubfile{\externaldocument{CGMPort}} % Get xrefs -- esp to appendix -- from main file; only works properly if main file has already been compiled; order: Compile this file, then main, then this one again
\begin{document}

\hfill{\tiny \texname.tex, \today}

\begin{verbatimwrite}{CGMPort.title}  % Write title to .title file
Cocco, Gomes, and Maenhout (2005) REMARK
\end{verbatimwrite}

\title{Cocco, Gomes, and Maenhout (2005) \\ REMARK}

\author{Mateo Vel\'asquez-Giraldo \and Matthew V. Zahn}

\keywords{Hic heac hoc}

\jelclass{XXX}

\maketitle 

\hypertarget{abstract}{}
\begin{abstract}
This paper contains the highlights from the REMARK file in Code>Python folder.
\end{abstract}

% Various resources 
\hypertarget{links}{}
\begin{small}
\parbox{\textwidth}{
\begin{center}
\begin{tabbing}
\texttt{~Archive:~} \= \= \url{http://econ.jhu.edu/people/ccarroll/BufferStockTheory.zip} \kill \\  %
\texttt{~~GitHub:~} \> \> \url{http://github.com/econ-ark/REMARK/REMARKS/CGMPort} \\
\texttt{~~~~~~~~~~} \> \> \textit{(In GitHub repo, see \texttt{/Code} for tools for solving and simulating the model)} \\
\end{tabbing}
\end{center}
          
\href{https://mybinder.org/v2/gh/matthew-zahn/CGMPort/develop?filepath=REMARK\%2FCGM_REMARK.ipynb}{CLICK HERE} for an interactive \href{https://en.wikipedia.org/wiki/Project\_Jupyter\#Jupyter_Notebook}{Jupyter Notebook} that uses the \href{https://econ-ark/HARK}{Econ-ARK/HARK} toolkit to produce our figures (warning: it may take several minutes to launch).  Information about citing the toolkit can be found at \href{https://econ-ark.org/acknowledging/}{Acknowleding Econ-ARK}.
} % end parbox{\textwidth}
\end{small}

\begin{authorsinfo}
\name{Contact: \href{mailto:mvelasq2@jhu.edu}{\texttt{mvelasq2@jhu.edu}}, Department of Economics, 590 Wyman Hall, Johns Hopkins University, Baltimore, MD 21218, \url{https://t.co/uaflostQyF?amp=1}.}
\name{Contact: \href{mailto:matthew.zahn@jhu.edu}{\texttt{matthew.zahn@jhu.edu}}, Department of Economics, 590 Wyman Hall, Johns Hopkins University, Baltimore, MD 21218, 
\url{http://matthewvzahn.com/}.}
\end{authorsinfo}

\thanks{All numerical results herein were produced using the \href{https://econ-ark/HARK}{Econ-ARK/HARK} toolkit; for further reference options see \href{https://econ-ark.org/acknowledging/}{Acknowleding Econ-ARK}.  Thanks to Chris Carroll and Sylvain Catherine for comments and guidance.}

\titlepagefinish

\newtheorem{defn}{Definition}
\newtheorem{theorem}{Theorem}

\hypertarget{Introduction}{}
\section{Introduction}


\hypertarget{The-Problem}{}
\section{The Problem}

\subsection{Setup}
\label{subsec:Setup}

The consumer solves an optimization problem from period
$t$ until the end of life at $T$ defined by the objective
\begin{verbatimwrite}{\EqDir/supfn.tex}
\begin{align}
  \label{eq:supfn}
  \max~ \Ex_{t}\left[\sum_{n=0}^{T-t} \DiscFac^{n} \uFunc(\cLevBF_{t+n})\right]
\end{align}
\end{verbatimwrite}
\begin{align}
  \label{eq:supfn}
  \max~ \Ex_{t}\left[\sum_{n=0}^{T-t} \DiscFac^{n} \uFunc(\cLevBF_{t+n})\right]
\end{align}

where
$\uFunc(\bullet)=\bullet^{1-\CRRA}/(1-\CRRA)$ is a constant relative
risk aversion utility function with $\CRRA > 1$.\footnote{The main
  results also hold for logarithmic utility which is the limit as
  $\CRRA \rightarrow 1$ but incorporating the logarithmic special case
  in the proofs is cumbersome and therefore
  omitted.}$^{,}$\footnote{We will define the infinite horizon
  solution as the limit of the finite horizon problem as the horizon
  $T-t$ approaches infinity.}  The consumer's initial condition is
defined by market resources $\mLevBF_{t}$ and permanent noncapital income $\pLevBF_{t}$.


\clearpage\vfill\eject

\onlyinsubfile{\bibliography{\econtexRoot/BufferStockTheory,economics}}

\end{document}
